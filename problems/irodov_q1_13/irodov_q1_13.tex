\documentclass{article}

\usepackage{amsmath, amssymb}
\usepackage{tikz}
\usepackage{subcaption}
\usepackage[margin=1.2in]{geometry}

\title{Irodov Problem 1.13}
\author{Rajarshi Dasgupta}

\begin{document}

\maketitle

\begin{quote}
	{ \bf 1.13 }
	Point A moves uniformly with velocity $v$
	so that the vector $\vec{v}$
	is continually `aimed' at point B
	which in its turn moves rectilinearly and uniformly
	with velocity $u < v$.
	At the initial moment of time $\vec{v} \perp \vec{u}$
	and the points are separated by a distance $l$.
	How soon will the points converge?
\end{quote}
I want to discuss the solution of the above problem
from `Problems in General Physics'
by I.E. Irodov [1]. % \cite{Irodov}.
Let $T$ be the time taken for points A and B to converge.
So, we need to find an expression for $T$
in terms of the quantities $u$, $v$, and $l$.

\section{School kid's solution}

\begin{figure}
		\centering
	\begin{subfigure}{0.45\linewidth}
		\centering
		\begin{tikzpicture}[scale=1.5]
				\node at (-1.2,2.5) {(a)};

				% Draw points A and B
				\filldraw (0,0) circle (1pt) node[below] {A};
				\filldraw (0,2) circle (1pt) node[above] {B};
				
				% Draw initial velocity vectors
				\draw[->] (0,0) -- (0,1) node[right] {$\vec{v}$};
				\draw[->] (0,2) -- (0.8,2) node[below] {$\vec{u}$};
				
				% Draw distance line
				\draw[<->] (-0.5,0) -- (-0.5,2) node[midway, left] {$l$};
		\end{tikzpicture}
	\end{subfigure}
	\begin{subfigure}{0.05\linewidth}
	\end{subfigure}
	\begin{subfigure}{0.45\linewidth}
		\centering
		\begin{tikzpicture}[scale=1.5]
				\node at (-1.2,2.5) {(b)};

				% Draw points A and B at t = 0
				\filldraw (0,0) circle (1pt) node[below] {A(0)};
				\filldraw (0,2) circle (1pt) node[above] {B(0)};
				
				% Draw points A and B at t < T
				\filldraw (0.8,1.5) circle (1pt) node[below] {A($t$)};
				\filldraw (1.2,2) circle (1pt) node[above] {B($t$)};
				
				% Draw distance line at t = 0
				\draw[<->] (-0.5,0) -- (-0.5,2) node[midway, left] {$l$};

				% Draw distance line at t
				\draw[<->] (0.8,1.5) -- (1.2,2);
				\node at (1.15,1.7) {$r$};

				% Angle between u and v 
				\draw[->] (1.2,2) ++(-0.4,0) arc (0:52:-0.4);
				\draw[dashed] (1.2,2) -- (0.7,2);
				\node at (0.7,1.8) {$\theta$};
		\end{tikzpicture}
	\end{subfigure}
	\caption{\label{basic_diag}
		(a)
			Velocities of points A and B
			separated by a distance of $l$
			at the initial moment of time $t = 0$
		(b)
			Velocities of points A and B
			separated by a distance of $r$
			at time $t \in (0,T)$.
		}
\end{figure}

Let $r$ be the distance between the points A and B,
and $\theta$ be the angle
between the velocities $\vec{v}$ and $\vec{u}$.
We consider
the relative velocity of A with respect to B
along AB and orthogonal to AB
to obtain the following relation.
\begin{align}
	\dot{r} &= -v + u \cos \theta \label{rdot} \\
	r \dot{\theta} &= -u \sin \theta \label{thetadot}
\end{align}

First,
we will eliminate $\dot{\theta}$
by differentiating equation \ref{rdot}
with respect to time
and then substituting the expression for $\dot{\theta}$
from equation \ref{thetadot}.
\begin{align*}
	& \ddot{r} = - u \dot{\theta} \sin \theta \\
	\implies & r \ddot{r} = u^2 \sin^2 \theta
\end{align*}
Now, we will eliminate $\theta$ from the above expression
by using the trigonometric identity
$\cos^2 \theta + \sin^2 \theta = 1$,
and equation \ref{rdot}.
\begin{align*}
	r \ddot{r} &= u^2 \sin^2 \theta \\
	&= u^2 (1 - \cos^2 \theta) \\
	&= u^2 \{1 - \left[\frac{\dot{r}+v}{u}\right]^2 \}
\end{align*}
Simplifying the above expression
we obtain the following second order differential equation.
\begin{align*}
	& r\ddot{r} + \dot{r}^2 + 2\dot{r}v + v^2 - u^2 = 0 \\
	\implies & \frac{d}{dt}(r\dot{r} + 2rv) + v^2 - u^2 = 0 \\
\end{align*}
Now, integrating with respect to time from $t = 0$ to $T$
we get the following expression.
\begin{align*}
	& \int_{t=0}^{t=T} \frac{d}{dt}(r\dot{r} + 2rv) dt
	+ \int_{t=0}^{t=T} (v^2 - u^2) dt = 0 \\
	\implies & r(T)\dot{r}(T) + 2r(T)v - r(0)\dot{r}(0) - 2r(0)v
	+ (v^2 - u^2)T = 0 \\
\end{align*}
We will use the fact that
\begin{align*}
	r(0) &= l \\
	\dot{r}(0) &= -v + u \cos \pi/2  & \mbox{(from equation \ref{rdot})}\\
	&= -v \\
	r(T) &= 0 & \mbox{(since A and B converge at time T)}
\end{align*}
to get the final expression for $T$.
\begin{align}
	T = \frac{lv}{v^2 - u^2}
\end{align}

\section{Physics major's solution}
To be written please wait

\section{Numerical solution}
To be written please wait

\section*{Discussion}
To be written please wait

\begin{thebibliography}{9}
\bibitem{Irodov} I. E. Irodov, \textit{Problems in General Physics}, Mir Publishers, 1988.
\end{thebibliography}

\end{document}
