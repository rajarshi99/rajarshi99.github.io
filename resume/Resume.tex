\documentclass{article}

\usepackage[misc]{ifsym}
\usepackage{fontawesome}
\usepackage{titlesec}
\usepackage{graphicx}
\usepackage[margin=2.5cm]{geometry}

\newcommand{\thetitle}{R\'esum\'e}
\newcommand{\theauthor}{Rajarshi Dasgupta}

\pagenumbering{gobble}

\titleformat{\section}
{\huge}
{}
{0cm}
{}[\titlerule]

\titleformat{\subsection}[runin]
{\bfseries  \vspace{-0.2cm}}
{}
{0cm}
{}

\begin{document}

\begin{center}
	{\bf \huge \theauthor}

	\vspace{1em}

	Indian Institute of Science, Bangalore\\
	\Letter \, rajarshi99@gmail.com \,
	\faMobilePhone \, 8427477104
\end{center}

\section{Research Interests}

Physics Informed Neural Networks,
Finite Element Methods,
Graph Neural Networks,
Complex Systems

\section{Experience}

\subsection{December 2024}
Organising Committee Member, CASML-2024, Indian Institute of Science (IISc), Bengaluru,
India's first scientific machine learning conference, hosting up to 300 participants

\subsection{October 2024}
Presented a poster on `Challenges with hp-VPINNs'
at the Indo-German Workshop on Hardware-aware Scientific Computing (IGHASC)

\subsection{May 2021 - May 2022}
Research on a dynamical system
exhibiting self organised criticality
in the presence of random links
under the guidance of Prof.~Sudeshna Sinha
at Indian Institute of Science Education and Research (IISER)
Mohali

% \subsection{June-July 2021}
% Teaching assistance
% for second year undergraduate course
% on Electrodynamics
% with instructor Dr.~Abhishek Chaudhuri;
% took problem solving sessions
% and taught the integral and differential forms
% of the Maxwell's equations

% \subsection{May-July 2019}
% Studied optimization methods
% and worked on developing an efficient variation of
% the feasible direction method
% also known as the Zoutendijk's method
% under the guidance of Prof.~Bhaskar Dasgupta

\subsection{May-July 2018}

Computational work on one dimensional maps,
understanding dimensionality of fractals,
learnt and implemented the `small-world' network,
studied spatio-temporal characteristics
of coupled lattice maps
under the guidance of Prof.~Sudeshna Sinha

\section{Education}

\subsection{Ongoing M.Tech. (Research)}
Pursuing my M.Tech.(Research) degree at the
department of Computational and Data Sciences, IISc
under the supervision of Prof.~Sashi Kumaar Ganesan
with a CPI of 8.1.

\subsection{BS-MS}
Physics Major and Data Science Minor
at IISER Mohali with a CPI of 8.9 out of 10.

\subsection{Class 12} 
Completed in Kendriya Vidyalaya, IIT Kanpur
with an average of 93\%.

\section{Publications}

\begin{itemize}
  \item
Dasgupta, R., Arun, A. \& Sinha, S. Emergent activity networks in a model of punctuated equilibrium. Eur. Phys. J. Plus 137, 1366 (2022). https://doi.org/10.1140/epjp/s13360-022-03581-y
\end{itemize}

\section{Skills}

\subsection{Programming Languages}
C, C++, Python, MATLAB

\subsection{Software}
\LaTeX,
shell scripting with basic UNIX programs
including AWK scripting

%\section{Relevant Courses}
%Computational methods in physics I;
%Nonlinear Dynamics, chaos and Complex System;
%Modelling Complex Systems

\section{Personal Information}

\subsection{Languages}
I speak Hindi, English
and have an understanding of Bengali.
\subsection{Interests}
Theatre and literature
% \subsection{DoB}
% 26 April 1999

\end{document}
