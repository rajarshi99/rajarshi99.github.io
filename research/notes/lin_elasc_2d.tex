\documentclass{article}

\usepackage{amsmath, amssymb}
%\usepackage[margin=1.2in]{geometry}

\newcommand{\pder}[2]{\frac{\partial #1}{\partial #2}}
\title{2D Linear Elasticity}

\author{
	Rajarshi Dasgupta
	}

\begin{document}

\maketitle

In the two dimensional elasticity problem
we need to find the displacement $u = u_x \hat{e}_x + u_y \hat{e}_y$
at a point $(x,y) \in \Omega$, the domain.
Extrapolation to two dimensions
of the simplistic one dimensional strain $= \Delta l / l$,
leads to a 2nd order strain tensor.

\begin{align*}
  & \epsilon = 
  \begin{bmatrix}
    \epsilon_{xx} & \epsilon_{xy} \\ \epsilon_{yx} & \epsilon_{yy}
  \end{bmatrix} \\
  & \epsilon_{ij} = \frac{1}{2} \left( \pder{u_i}{j} + \pder{u_j}{i} \right)
  & i,j \mbox{ are placeholders for } x,y\\
  \implies & \epsilon_{xx} = \pder{u_x}{x} \\
  & \epsilon_{xy} = \left( \pder{u_x}{y} + \pder{u_y}{x} \right)/2\\
  & \epsilon_{yx} = \left( \pder{u_y}{x} + \pder{u_x}{y} \right)/2\\
  & \epsilon_{yy} = \pder{u_y}{y} \\
\end{align*}

The constitutive relation is linear
and is defined by a 4th order tensor $C$.
So, the 2nd order stress tensor is given by
\begin{align*}
  \sigma_{ij} = \sum_{k=x,y} \sum_{l=x,y} C_{ijkl} \epsilon_{kl}
\end{align*}
where again $i$ and $j$ are placeholders for $x$ or $y$.
Let $\hat{n}$ be the outward normal on a point on the boundary
of the domain $\Omega$.
Now, the components of the traction vector is given by
\begin{align*}
  t_i & = \sum_{j=x,y} \sigma_{ij} n_j & i=x,y
\end{align*}
The essential boundary condition
would be to specify $u$,
wheras the natural boundary condition
would be to specify the traction $t$.

To make our lives easier we consider the engineering strain
$[\epsilon_{xx} \, \epsilon_{yy} \, \gamma_{xy}]^\mathsf{T}$
where $\gamma_{xy} = 2 \epsilon_{xy}$ is engineering shear strain.
Now, we can write the constitutive relation using a matrix $C$.
\begin{align*}
  \begin{bmatrix}
    \sigma_{xx} \\ \sigma_{yy} \\ \sigma_{xy}
  \end{bmatrix}
  = C
  \begin{bmatrix}
    \epsilon_{xx} \\ \epsilon_{yy} \\ \gamma_{xy}
  \end{bmatrix}
\end{align*}
Two of the usual choices for $C$ are
\begin{align*}
  \frac{E}{1 - \nu^2}
  \begin{bmatrix}
    1 & \nu & 0 \\
    \nu & 1 & 0 \\
    0 & 0 & \frac{1-\nu}{2} \\
  \end{bmatrix}, &&
  \frac{E}{(1+\nu)(1-2\nu)}
  \begin{bmatrix}
    1-\nu & \nu & 0 \\
    \nu & 1-\nu & 0 \\
    0 & 0 & \frac{1-2\nu}{2} \\
  \end{bmatrix}
\end{align*}
where $E$ is the Youngs modulus
and $\nu$ is the Poisson's ratio.

Now that we have clearly defined
the stress and strain we can write the governing equation
\begin{align*}
  & \pder{\sigma_{xx}}{x} + \pder{\sigma_{xy}}{y} + b_x = 0 \\
  & \pder{\sigma_{xy}}{x} + \pder{\sigma_{yy}}{y} + b_y = 0 \\
\end{align*}
where $b_x \hat{e}_x + b_y \hat{e}_y$ is the given body force
at the point $(x,y)$.



\end{document}

